\section{Installation}
\label{installation}
\index{installation}
The \AAD module requires Python 2.4 or later. It was developed on linux\index{linux}
(SUSE\index{SUSE}\index{linux!SUSE},
fedora\index{fedora}\index{linux!fedora}), Solaris 5.8\index{Solaris} and
MacOSX\index{MacOSX}, however there should be no problems installing it on any machine
hosting Python.

Individual functions, such as the transformation to numarray\index{numarray},
numpy\index{numpy} or the FITS\index{FITS} format, obviously require the python
modules for the formats they convert to. But there is no general need to install them.\\

The current version 1.1 of \AAD is distributed as the source
archive {\tt asciidata-1.1.tar.gz} from the \AAD webpage
\index{AstroAsciiData webpage} at\newline
\htmladdnormallink{http://www.stecf.org/software/PYTHONtools/astroasciidata/}{http://www.stecf.org/software/PYTHONtools/astroasciidata/}.
Installing the module is not difficult. Unpack the tarball with:
\begin{verbatim}
> gunzip asciidata-1.1.tar.gz
> tar -xvf asciidata-1.1.tar
\end{verbatim}

Then enter the the unpacked directory and do the usual:
\begin{verbatim}
> cd asciidata-1.1
> python setup.py install
\end{verbatim}

After installation, some Unit Tests can be executed with:
\begin{verbatim}
> python setup.py test
\end{verbatim}
If there are no errors reported in the Unit Tests, the proper
working  of the module is assured. Failed tests may happen due to
missing pyhon modules (numpy, PyFITS or numarray) and can be neglected
if you do not intend to convert into these formats.\\

In all classes and sub-modules the epydoc-conventions have been used in
the inline documentation. In case that {\tt epydoc}\index{epydoc}
(\htmladdnormallink{http://epydoc.sourceforge.net/}{http://epydoc.sourceforge.net/})
is installed, the command
\begin{verbatim}
> epydoc Lib/
\end{verbatim}
creates webpages from the inline documenatation, which are written to the
the directory './html'. This would be certainly a very good start for users
who really want to find out what is behind the module or intend to subclass
it to e.g. support their own, custom made ASCII table format with column
names. In case that you just want to use the \AAD module, there is no
need to look at its inline documentation.

\subsection{Lemma for version 1.1.1}
This newest version is rather a patch release. The advent of numpy-1.1 require
some small changes not even to the code itself, but to the unit tests. Since these
are integral part of the software release, we decided to upgrade and call the new bundle
version 1.1.1.

\subsection{Release notes for version 1.1}
Except for minor bug fixes, version 1.1 contains the following improvements:
\begin{itemize}
  \item export to
  \htmladdnormallink{numpy}{http://numpy.scipy.org/}\index{numpy};
  \item export to
  \htmladdnormallink{FITS}{http://fits.gsfc.nasa.gov/fits_home.html}\index{FITS} via numpy;
  \item some convenient functions were added (see e.g. Sect.\ref{adm_strip});
  \item column names can contain basic arithmetic operators now.
  \item code with deprecation warnings was replaced.
\end{itemize}
