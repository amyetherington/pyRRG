\section{The detailed description}
\label{details}
The \AAD module was developed in Python using an Object Oriented (OO) approach
with classes and methods. This can not be hidden in the usage of the
\AAD module.
Working with \AAD means creating its class objects, accessing the class data
and executing class methods. This might be confusing for users who are
not familiar with this terminology and its meaning.

However this manual makes no attempt to introduce the OO terminology,
and its complete understanding is not really necessary in order to use
the \AAD module . The user can simply stick to
a strictly {\it phenomenological} approach by looking at the examples
and transferring them to his/her own applications. Nevertheless the OO terms
are used to structure this section of the manual.

%
% Section for functions
%
\subsection{Functions}
\label{functions}
\index{functions}
The \AAD module contains the two functions {\tt open()} and {\tt create()}.
These function serve as a starting point for the work with ASCII tables,
since both return an \ad object by either opening  and loading
an existing ASCII file using {\tt open()} or creating an empty \ad object
from scratch with {\tt create()}.

\subsubsection{open()}
\label{functions_open}
\index{functions!open()}
This function loads an existing ASCII table file. An \ad object is
created and the data stored in the ASCII table is transferred to the
AsciiData object. Various function parameters specify e.g. the
character used as a delimiter to separate adjacent column elements.

\prgrf{Usage}
open(filename, null=None, delimiter=None, comment=None)

\prgrf{Parameters}
\begin{tabular}{lcl}
filename &{\it string}& the name of the ASCII table\\
         &            &  file to be loaded\\
null     &{\it string}& the character/string representing\\
         &            & a null-entry\\
delimiter&{\it string}& the delimiter separating the\\
         &            & columns \\
comment  &{\it string}& the character/string indicating\\
         &            & a comment\\
\end{tabular}

\prgrf{Return}
- an \ad object

\prgrf{Examples}
\begin{enumerate}
\item Load the file 'example.txt' and print the result. The file 'example.txt
looks like:
\begin{small}
\begin{verbatim}
#
# Some objects in the GOODS field
#
unknown  189.2207323  62.2357983  26.87  0.32
 galaxy  189.1408929  62.2376331  24.97  0.15
   star  189.1409453  62.1696844  25.30  0.12
 galaxy  188.9014716  62.2037839  25.95  0.20
\end{verbatim}
\end{small}
The command sequence is:
\begin{small}
\begin{verbatim}
>>> example = asciidata.open('example.txt')
>>> print example
#
# Some objects in the GOODS field
#
unknown  189.2207323  62.2357983  26.87  0.32
 galaxy  189.1408929  62.2376331  24.97  0.15
   star  189.1409453  62.1696844  25.30  0.12
 galaxy  188.9014716  62.2037839  25.95  0.20
\end{verbatim}
\end{small}

\item Load the file 'example2.txt' and print the results. 'example2.txt':
\begin{small}
\begin{verbatim}
@
@ Some objects in the GOODS field
@
unknown $ 189.2207323 $ 62.2357983 $ 26.87 $ 0.32
 galaxy $      *      $ 62.2376331 $ 24.97 $ 0.15
   star $ 189.1409453 $ 62.1696844 $ 25.30 $  *
 *      $ 188.9014716 $     *      $ 25.95 $ 0.20
\end{verbatim}
\end{small}

Load and print:
\begin{small}
\begin{verbatim}
>>> example2 = asciidata.open('example2.txt', null='*', \
                              delimiter='$', comment='@')
>>> print example2
@
@ Some objects in the GOODS field
@
unknown  $  189.2207323 $  62.2357983 $  26.87 $  0.32
 galaxy  $            * $  62.2376331 $  24.97 $  0.15
   star  $  189.1409453 $  62.1696844 $  25.30 $     *
       * $  188.9014716 $           * $  25.95 $  0.20
\end{verbatim}
\end{small}

\end{enumerate}

\subsubsection{create()}
\label{functions_create}
\index{functions!create()}

\prgrf{Usage}
create(ncols, nrows, null=None, delimiter=None)

\prgrf{Parameters}
\begin{tabular}{lcl}
ncols    &{\it int}& number of columns to be created\\
nrows    &{\it int}& number of rows to be created\\
null     &{\it string}& the character/string representing a null-entry\\
delimiter&{\it string}& the delimiter separating the columns \\
\end{tabular}

\prgrf{Return}
- an \ad object

\prgrf{Examples}
\begin{enumerate}
\item Create an \ad object with 3 columns and 2 rows, print the result:
\begin{verbatim}
>>> example3 = asciidata.create(3,2)
>>> print (example3)
      Null       Null       Null
      Null       Null       Null
\end{verbatim}

\item As in 1., but use a different delimiter and NULL value, print the result:
\begin{verbatim}
>>> example4 = asciidata.create(3,2,delimiter='|', null='<*>')
>>> print (example4)
       <*> |        <*> |        <*>
       <*> |        <*> |        <*>
\end{verbatim}
\end{enumerate}

%
% Section the asciidata class
%
\subsection{The AsciiData class}
\label{adclass}
\index{classes}\index{classes!AsciiData}
\index{AsciiData class}
The \ad class is the central class in the \AAD module.
After creating \ad objects with one of the functions introduced in
Sect.\ \ref{functions}, the returned objects are modified using
its methods.

\subsubsection{AsciiData data}
\label{add}
\index{class data}\index{AsciiData!data}
\ad objects contain some information which is important to the user
and can be used in the processing. Although it is possible,
this class data should {\bf never} be changed directly by the user.
All book-keeping is done internally such that e.g. the value of
{\tt ncols} is adjusted when deleting a column.

\prgrf{Data}
\begin{tabular}{lcl}
filename &{\it string}& file name associated to the object\\
ncols    &{\it int}& number of columns\\
nrows    &{\it int}& number of rows\\
\end{tabular}

\prgrf{Examples}
\begin{enumerate}
\item Go over all table entries an store values:
\begin{small}
\begin{verbatim}
>>> example3 = asciidata.create(100,100)
>>> for cindex in range(example3.ncols):
...     for rindex in range(example3.nrows):
...             example3[cindex][rindex] = do_something(rindex, rindex)
...
\end{verbatim}
\end{small}
\item Derive a new filename and save the table to this filename:
\begin{verbatim}
>>> print example2.filename
example2.txt
>>> newname = example2.filename + '.old'
>>> print newname
example2.txt.old
>>> example2.writeto(newname)
\end{verbatim}
\end{enumerate}

\subsubsection{AsciiData method get}
\label{adm_get}
\index{methods}\index{methods!AsciiData!get}
\index{get}
%
This method retrieves list members of an \ad instance. These list members
are the \ac instances, which are accessed via their column name or
column number.\\
The method returns only the {\it reference} to the column, therefore
changing the returned \ac instance means also changing the original \ad
instance (see Example 2)! For a deep copy of an item the method
{\tt deepcopy()} \index{deep copy} in the python module {\tt copy}
(see \htmladdnormallink{www.python.org}{www.python.org}) or the
method {\tt copy} of the \ac class (see Sect. \ref{acm_copy}) must be
used instead.

\prgrf{Usage}
adata\_column = adata\_object[col\_spec]\\
{\it or}\\
adata\_column = operator.getitem(adata\_object, col\_spec)

\prgrf{Parameters}
\begin{tabular}{lcl}
col\_spec &{\it string/int}& column specification, either by column name or column number\\
\end{tabular}

\prgrf{Return}
- an \ac instance

\prgrf{Examples}
\begin{enumerate}
\item Retrieve the second column of the table:
\begin{small}
\begin{verbatim}
>>> print example
#
# most important sources!!
#
    1  1.0  red  23.08932 -19.34509
    2  9.5 blue  23.59312 -19.94546
    3  3.5 blue  23.19843 -19.23571
>>> aad_col = example[1]
>>> print aad_col
Column: column2
 1.0
 9.5
 3.5
>>>
\end{verbatim}
\end{small}
\item Retrieve the second column of the table. Demonstrate that only
a shallow copy (reference) is returned:
\begin{small}
\begin{verbatim}
>>> print example
#
# most important sources!!
#
    1  1.0  red  23.08932 -19.34509
    2  9.5 blue  23.59312 -19.94546
    3  3.5 blue  23.19843 -19.23571
>>> ad_col = operator.getitem(example, 'column1')
>>> print ad_col
Column: column1
    1
    2
    3
>>> ad_col[1] = 'new!'
>>> print ad_col
Column: column1
   1
new!
   3
>>> print example
#
# most important sources!!
#
   1  1.0  red  23.08932 -19.34509
new!  9.5 blue  23.59312 -19.94546
   3  3.5 blue  23.19843 -19.23571
>>>
\end{verbatim}
\end{small}
\end{enumerate}


\subsubsection{AsciiData method set}
\label{adm_set}
\index{methods}\index{methods!AsciiData!set}
\index{set}
%
This methods sets list members, which means columns, of an \ad instance.
The list member to be changed is addressed either via its column name or
the column number.\\
Obviously the replacing object must be an \ac instance which contains
an equal number of rows. Otherwise an exception is risen.

\prgrf{Usage}
adata\_object[col\_spec] = adata\_column\\
{\it or}\\
operator.setitem(adata\_object, col\_spec, adata\_column)


\prgrf{Parameters}
\begin{tabular}{lcl}
col\_spec &{\it string/int}& column specification, either by column name or column number\\
adata\_column &{\it AsciiColumn}& the \ac instance to replace the previous column\\
\end{tabular}

\prgrf{Return}
-

\prgrf{Examples}
\begin{enumerate}
\item Replace the third row of the table 'exa\_1' with the third row of table 'exa\_2'.
Please note the interplay between the {\tt get}- and the {\tt set}-method of
the \ad class:
\begin{small}
\begin{verbatim}
>>> exa_1 = asciidata.open('some_objects.cat')
>>> exa_2 = asciidata.open('some_objects_2.cat', delimiter='|', comment='@', null='*')
>>> print exa_1
#
# most important objects
#
    1  1.0  red  23.08932 -19.34509
    2  9.5 blue  23.59312 -19.94546
    3  3.5 blue  23.19843 -19.23571
>>> print exa_2
@
@
@
   10| 0.0|  pink | 130.3757| 69.87343
   25| 5.3| green | 130.5931| 69.89343
   98| 3.5|      *| 130.2984| 69.30948
>>> exa_1[2] = exa_2[2]
>>> print exa_1
#
# most important objects
#
    1  1.0   pink   23.08932 -19.34509
    2  9.5  green   23.59312 -19.94546
    3  3.5    Null  23.19843 -19.23571
>>>
\end{verbatim}
\end{small}
\end{enumerate}

\subsubsection{AsciiData method tofits()}
\label{adm_tofits}
\index{methods}\index{methods!AsciiData!tofits()}
\index{tofits()}
%
The method transforms an \ad instance to a fits-table extension\index{FITS}.
This extension might be used with other extensions to build a
multi-extension fits-file.\\
Please use the \ad method {\tt writetofits()}
(see Sect.\ \ref{adm_writetofits}) to make both, the conversions and
storing as a fits-file onto hard disk in one step.\\
The module \index{PyFITS}{\tt PyFITS} (see \htmladdnormallink{http://www.stsci.edu/resources/software\_hardware/pyfits}{http://www.stsci.edu/resources/software_hardware/pyfits})
must be installed to run this method. The transformation fails if the
\ad instance contains any {\tt Null} elements (due to a limitation
of the {\tt numarray} objects, which are essential for the method).


\prgrf{Usage}
aad\_object.tofits()

\prgrf{Parameters}
-

\prgrf{Return}
- a table fits extension

\prgrf{Examples}
\begin{enumerate}
\item Convert an \ad object to a fits-table extension and append it to an
already existing fits-table (the example is executed in PyRAF):
\begin{small}
\begin{verbatim}
--> catfits exa_table.fits
EXT#  FITSNAME      FILENAME              EXTVE DIMENS       BITPI OBJECT
0     exa_table.fit                                          16
1       BINTABLE    BEAM_1A                     14Fx55R            1
--> exa = asciidata.open('some_objects.cat')
--> tab_hdu = exa.tofits()
--> tab_all = pyfits.open('exa_table.fits', 'update')
--> tab_all.append(tab_hdu)
--> tab_all.close()
--> catfits exa_table.fits
EXT#  FITSNAME      FILENAME              EXTVE DIMENS       BITPI OBJECT
0     exa_table.fit                                          16
1       BINTABLE    BEAM_1A                     14Fx55R            1
2       BINTABLE                                5Fx3R
-->
\end{verbatim}
\end{small}
\end{enumerate}

\subsubsection{AsciiData method writetofits()}
\label{adm_writetofits}
\index{methods}\index{methods!AsciiData!writetofits()}
\index{writetofits()}
%
The method transforms an \ad instance to a fits-table\index{FITS} and stores
the fits-table to the disk. The filename is either specified as a parameter
or is derived from the filename of the original ascii-table. In the latter
case the file extension is changed '.fits'\\
The module \index{PyFITS}{\tt PyFITS} (see \htmladdnormallink{http://www.stsci.edu/resources/software\_hardware/pyfits}{http://www.stsci.edu/resources/software_hardware/pyfits})
must be installed to run this method. The transformation fails if the
\ad instance contains any {\tt Null} elements (due to a limitation
of the {\tt numarray} objects, which are essential for the method).

\prgrf{Usage}
aad\_object.writetofits(fits\_name=None)


\prgrf{Parameters}
\begin{tabular}{lcl}
fits\_name & {\it string} & the name of the fits-file\\
\end{tabular}

\prgrf{Return}
- the fits file to which the \ad instance was written

\prgrf{Examples}
\begin{enumerate}
\item Store an \ad instance as a fits-file, using the default name:
\begin{small}
\begin{verbatim}
test>ls
some_objects.cat
test>python
Python 2.4.2 (#1, Nov 10 2005, 11:34:38)
[GCC 3.3.3 20040412 (Red Hat Linux 3.3.3-7)] on linux2
Type "help", "copyright", "credits" or "license" for more information.
>>> import asciidata
>>> exa = asciidata.open('some_objects.cat')
>>> fits_name = exa.writetofits()
>>> fits_name
'some_objects.fits'
>>>
test>ls
some_objects.cat  some_objects.fits
test>
\end{verbatim}
\end{small}


\item Store an \ad instance to the fits-file 'test.fits':
\begin{small}
\begin{verbatim}
test>ls
some_objects.cat
test>python
Python 2.4.2 (#1, Nov 10 2005, 11:34:38)
[GCC 3.3.3 20040412 (Red Hat Linux 3.3.3-7)] on linux2
Type "help", "copyright", "credits" or "license" for more information.
>>> import asciidata
>>> exa = asciidata.open('some_objects.cat')
>>> fits_name = exa.writetofits('test.fits')
>>> fits_name
'test.fits'
>>>
test>ls
some_objects.cat  test.fits
test>
\end{verbatim}
\end{small}
\end{enumerate}


\subsubsection{AsciiData method writetohtml()}
\label{adm_writetohtml}
\index{methods}\index{methods!AsciiData!writetohtml()}
\index{writetohtml()}
%
The method writes the data of an \ad instance formatted as the content of
an html-table to the disk. Strings used as attributes can be specified
for the tags {\tt<tr>} and {\tt <td>}. The name of the html-file is either
given as parameter or is derived from the name of the original ascii-table.
In the latter case the file extension is changed '.html'.\\
The html-table is neither opened nor closed at
the beginning and end of the file, respectively. Also column names and
other meta information is NOT used in the html.

\prgrf{Usage}
aad\_object.writetohtml(html\_name=None, tr\_attr=None, td\_attr=None)

\prgrf{Parameters}
\begin{tabular}{lcl}
html\_name & {\it string} & the name of the html-file\\
tr\_attr & {\it string} & attribute string for the tr-tag\\
td\_attr & {\it string} & attribute string for the td-tag\\
\end{tabular}

\prgrf{Return}
- the name of the html-file

\prgrf{Examples}
\begin{enumerate}
\item Write an \ad instance to an html-file:
\begin{small}
\begin{verbatim}
>>> exa = asciidata.open('some_objects.cat')
>>> exa.writetohtml()
'some_objects.html'
>>>
test>more 'some_objects.html'
<tr><td>    1</td><td> 1.0</td><td> red</td><td> 23.08932</td><td>-19.34509</td></tr>
<tr><td>    2</td><td> 9.5</td><td>blue</td><td> 23.59312</td><td>-19.94546</td></tr>
<tr><td>    3</td><td> 3.5</td><td>blue</td><td> 23.19843</td><td>-19.23571</td></tr>
test>
\end{verbatim}
\end{small}
\item Write an \ad instance to the html-file 'mytab.tab', using attributes
for the tags:
\begin{small}
\begin{verbatim}
>>> exa = asciidata.open('some_objects.cat')
>>> html_name = exa.writetohtml('mytab.tab',tr_attr='id="my_tr"',td_attr='bgcolor="RED"')
>>> print html_name
mytab.tab
>>>
test>more mytab.tab
<tr id="my_tr"><td bgcolor="RED">    1</td><td bgcolor="RED"> 1.0</td><td bgcolor="RED">
 red</td><td bgcolo
r="RED"> 23.08932</td><td bgcolor="RED">-19.34509</td></tr>
<tr id="my_tr"><td bgcolor="RED">    2</td><td bgcolor="RED"> 9.5</td><td bgcolor="RED">
blue</td><td bgcolo
r="RED"> 23.59312</td><td bgcolor="RED">-19.94546</td></tr>
<tr id="my_tr"><td bgcolor="RED">    3</td><td bgcolor="RED"> 3.5</td><td bgcolor="RED">
blue</td><td bgcolo
r="RED"> 23.19843</td><td bgcolor="RED">-19.23571</td></tr>
test>
\end{verbatim}
\end{small}
\end{enumerate}


\subsubsection{AsciiData method writetolatex()}
\label{adm_writetolatex}
\index{methods}\index{methods!AsciiData!writetolatex()}
\index{writetolatex()}
%
The method writes the data of an \ad instance formatted as the content of
a latex-table to the disk. The name of the html-file is either
given as parameter or is derived from the name of the original ascii-table.
In the latter case the file extension is changed '.tex'.

\prgrf{Usage}
aad\_object.writetolatex(latex\_name=None)


\prgrf{Parameters}
\begin{tabular}{lcl}
latex\_name & {\it string} & the name of the latex-file\\
\end{tabular}

\prgrf{Return}
- the name of the latex-file

\prgrf{Examples}
\begin{enumerate}
\item Write the content of an \ad instance to 'latextab.tb':
\begin{small}
\begin{verbatim}
>>> exa = asciidata.open('some_objects.cat')
>>> latex_name = exa.writetolatex('latex.tb')
>>> print latex_name
latex.tb
>>>
test>more latex.tb
    1& 1.0& red& 23.08932&-19.34509\\
    2& 9.5&blue& 23.59312&-19.94546\\
    3& 3.5&blue& 23.19843&-19.23571\\
test>
\end{verbatim}
\end{small}
\end{enumerate}


\subsubsection{AsciiData method sort()}
\label{adm_sort}
\index{methods}\index{methods!AsciiData!sort()}
\index{sort()}
%
This method sorts the data in an \ad instance according to the values
in a specified column. Sorting in ascending and descending order is
possible.\\
There are two different sorting algorithms implemented. A very fast algorithm
can be used for making a single, 'isolated' sort process. If the desired result
of the sort process can only be reached with consequtive sortings on different
columns, a slower algorithm must be used which does not introduce random
swaps of rows (see the Examples and Sect.??? for details).


\prgrf{Usage}
adata\_object.sort(colname, descending=0, ordered=0)

\prgrf{Parameters}
\begin{tabular}{lcl}
colname & {\it string/integer} & the specification of the sort column\\
descending & {\it integer} & sort in ascending ($=0$) or descending ($=1$) order\\
ordered & {\it integer} & use the fast ($=0$) algorithm or the slow
($=1$) which avoids unnecessary row swaps\\
\end{tabular}

\prgrf{Return}
-

\prgrf{Examples}
\begin{enumerate}
\item Sort a table in ascending order of the values in the second column:
\begin{small}
\begin{verbatim}
>>> sort = asciidata.open('sort_objects.cat')
>>> print sort
    1     0     1     1
    2     1     0     3
    3     1     2     4
    4     0     0     2
    5     1     2     1
    6     0     0     3
    7     0     2     4
    8     1     1     2
    9     0     1     5
   10     1     2     6
   11     0     0     6
   12     1     1     5
>>> sort.sort(1)
>>> print sort
    1     0     1     1
    6     0     0     3
    9     0     1     5
   11     0     0     6
    7     0     2     4
    4     0     0     2
   12     1     1     5
    2     1     0     3
   10     1     2     6
    3     1     2     4
    5     1     2     1
    8     1     1     2
>>>
\end{verbatim}
\end{small}

\item Use the result from example 1, and sort the table in descending
order of the first column:
\begin{small}
\begin{verbatim}
>>> sort.sort(0, descending=1)
>>> print sort
   12     1     1     5
   11     0     0     6
   10     1     2     6
    9     0     1     5
    8     1     1     2
    7     0     2     4
    6     0     0     3
    5     1     2     1
    4     0     0     2
    3     1     2     4
    2     1     0     3
    1     0     1     1
>>>
\end{verbatim}
\end{small}

\item Sort the table first along column 3 and then along column 2. The
resulting table is sorted along column 2, but in addition it is
ordered along column 3 for equal values in column 2.
This works only using the slower, ordered sorting algorithm:
\begin{small}
\begin{verbatim}
>>> sort.sort(2, ordered=1)
>>> sort.sort(1, ordered=1)
>>> print sort
   11     0     0     6
    6     0     0     3
    4     0     0     2
    9     0     1     5
    1     0     1     1
    7     0     2     4
    2     1     0     3
   12     1     1     5
    8     1     1     2
   10     1     2     6
    5     1     2     1
    3     1     2     4
>>>
\end{verbatim}
\end{small}
\item As the previous example, but using the faster, un-ordered sorting
algorithm. Generally the values are not sorted according  to column 3 if
equal in column 2:
\begin{small}
\begin{verbatim}
>>> sort.sort(2, ordered=0)
>>> sort.sort(1, ordered=0)
>>> print sort
    1     0     1     1
    4     0     0     2
    7     0     2     4
   11     0     0     6
    9     0     1     5
    6     0     0     3
   12     1     1     5
    2     1     0     3
    3     1     2     4
   10     1     2     6
    5     1     2     1
    8     1     1     2
>>>
\end{verbatim}
\end{small}
\end{enumerate}

\subsubsection{AsciiData method len()}
\label{adm_len}
\index{methods}\index{methods!AsciiData!len()}
\index{len()}
%
This method defines a length for every \ad instance, which is the number
of columns.

\prgrf{Usage}
len(aad\_object)

\prgrf{Parameters}
-

\prgrf{Return}
- the length of the \ad instance

\prgrf{Examples}
\begin{enumerate}
\item Determine and print the length of an \ad instance:
\begin{small}
\begin{verbatim}
>>> exa = asciidata.open('some_objects.cat')
>>> print exa
#
# most important objects
#
    1  1.0  red  23.08932 -19.34509
    2  9.5 blue  23.59312 -19.94546
    3  3.5 blue  23.19843 -19.23571
>>> length = len(exa)
>>> print length
5
>>>
\end{verbatim}
\end{small}
\end{enumerate}

\subsubsection{AsciiData iterator type}
\label{adm_iterator}
\index{methods}\index{methods!AsciiData!iterator}
\index{iterator}
%
This defines an iterator over an \ad instance. The iteration is finished after
{\tt aad\_object.ncols} calls and returns each column in subsequent calls.
Please not that it is {\bf not} possible to change these columns.
\prgrf{Usage}
for iter in aad\_object:\\
... $<do\ something>$

\prgrf{Parameters}
-

\prgrf{Return}
-

\prgrf{Examples}
\begin{enumerate}
\item Iterate over an \ad instance and print each column name:
\begin{small}
\begin{verbatim}
>>> exa = asciidata.open('sort_objects.cat')
>>> for col in exa:
...     print col.colname
...
column1
column2
column3
column4
>>>
\end{verbatim}
\end{small}
\end{enumerate}


\subsubsection{AsciiData method append()}
\label{adm_append}
\index{methods}\index{methods!AsciiData!append()}
\index{append()}
Invoking this method is the formal way to append an new column
to and \ad object. When created there are only {\tt Null} entries in the
new column. The alternative way is just to specify a column
with an unknown name (see Sect.\ \ref{existingdata}).

\prgrf{Usage}
adata\_object.append(col\_name)

\prgrf{Parameters}
\begin{tabular}{lcl}
col\_name &{\it string}& the name of the new column\\
\end{tabular}

\prgrf{Return}
- the number of the columns created

\prgrf{Examples}
\begin{enumerate}
\item Append a new column {\tt 'newcolumn'} to the \ad object:
\begin{small}
\begin{verbatim}
>>> print example2
@
@ Some objects in the GOODS field
@
unknown  $  189.2207323 $  62.2357983 $  26.87 $  0.32
 galaxy  $            * $  62.2376331 $  24.97 $  0.15
   star  $  189.1409453 $  62.1696844 $  25.30 $     *
       * $  188.9014716 $           * $  25.95 $  0.20
>>> cnum = example2.append('newcolumn')
>>> print cnum
5
>>> print example2
@
@ Some objects in the GOODS field
@
unknown  $  189.2207323 $  62.2357983 $  26.87 $  0.32 $          *
 galaxy  $            * $  62.2376331 $  24.97 $  0.15 $          *
   star  $  189.1409453 $  62.1696844 $  25.30 $     * $          *
       * $  188.9014716 $           * $  25.95 $  0.20 $          *
\end{verbatim}
\end{small}
\end{enumerate}


\subsubsection{AsciiData method str()}
\label{adm_str}
\index{methods}\index{methods!AsciiData!str()}
\index{str()}
This methods converts the whole \ad object into a string.
Columns are separated with the delimiter, empty elements are represented
by the Null-string and the header is indicated by a comment-string at
the beginning. In this method the class object appears as a function
argument and the method call is different from the usual form such as
in Sect.\ \ref{adm_append}

\prgrf{Usage}
str(adata\_object)

\prgrf{Parameters}
-

\prgrf{Return}
- the string representing the \ad object

\prgrf{Examples}
\begin{enumerate}
\item Print an \ad object to the screen:
\begin{verbatim}
>>> print str(example2)
@
@ Some objects in the GOODS field
@
unknown  $  189.2207323 $  62.2357983 $  26.87 $  0.32
 galaxy  $            * $  62.2376331 $  24.97 $  0.15
   star  $  189.1409453 $  62.1696844 $  25.30 $     *
       * $  188.9014716 $           * $  25.95 $  0.20
\end{verbatim}
\item Store the sting representation of an \ad object:
\begin{verbatim}
>>> big_string = str(example2)
>>> print big_string
@
@ Some objects in the GOODS field
@
unknown  $  189.2207323 $  62.2357983 $  26.87 $  0.32
 galaxy  $            * $  62.2376331 $  24.97 $  0.15
   star  $  189.1409453 $  62.1696844 $  25.30 $     *
       * $  188.9014716 $           * $  25.95 $  0.20
\end{verbatim}
\end{enumerate}

\subsubsection{AsciiData method del}
\label{adm_del}
\index{methods}\index{methods!AsciiData!del}
\index{del}
This method deletes a column specified either by its name or
by the column number. Also this method call is slightly
different from the usual form such as in Sect.\ \ref{adm_append}
or \ref{adm_delete}.

\prgrf{Usage}
del adata\_obj[col\_spec]

\prgrf{Parameters}
\begin{tabular}{lcl}
col\_spec &{\it string/int}& column specification either by name or by
the column number\\
\end{tabular}

\prgrf{Return}
-

\prgrf{Examples}
\begin{enumerate}
\item Delete the column with name 'column1':
\begin{verbatim}
>>> print example2
@
@ Some objects in the GOODS field
@
unknown  $  189.2207323 $  62.2357983 $  26.87 $  0.32
 galaxy  $            * $  62.2376331 $  24.97 $  0.15
   star  $  189.1409453 $  62.1696844 $  25.30 $     *
       * $  188.9014716 $           * $  25.95 $  0.20
>>> del example2['column5']
>>> print example2
@
@ Some objects in the GOODS field
@
unknown  $  189.2207323 $  62.2357983 $  26.87
 galaxy  $            * $  62.2376331 $  24.97
   star  $  189.1409453 $  62.1696844 $  25.30
       * $  188.9014716 $           * $  25.95
\end{verbatim}
\item Delete the second column:
\begin{verbatim}
>>> print example2
@
@ Some objects in the GOODS field
@
unknown  $  189.2207323 $  62.2357983 $  26.87 $  0.32
 galaxy  $            * $  62.2376331 $  24.97 $  0.15
   star  $  189.1409453 $  62.1696844 $  25.30 $     *
       * $  188.9014716 $           * $  25.95 $  0.20
>>> del example2[1]
>>> print example2
@
@ Some objects in the GOODS field
@
unknown  $  62.2357983 $  26.87 $  0.32
 galaxy  $  62.2376331 $  24.97 $  0.15
   star  $  62.1696844 $  25.30 $     *
       * $           * $  25.95 $  0.20
\end{verbatim}
\end{enumerate}

\subsubsection{AsciiData method delete()}
\label{adm_delete}
\index{methods}\index{methods!AsciiData!delete()}
\index{delete()}
This method deletes rows in an \ad object. The rows to be deleted
are specified in the parameters.

\prgrf{Usage}
adata\_obj.delete(start, end)

\prgrf{Parameters}
\begin{tabular}{lcl}
start &{\it int}& the first row to be deleted\\
end   &{\it int}& the first row {\bf not} to be deleted\\
\end{tabular}

\prgrf{Return}
-

\prgrf{Examples}
\begin{enumerate}
\item Delete the row with index 1:
\begin{verbatim}
>>> print example2
@
@ Some objects in the GOODS field
@
unknown   $  189.2207323 $  62.2357983 $  26.87 $  0.32
 galaxy   $            * $  62.2376331 $  24.97 $  0.15
   star   $  189.1409453 $  62.1696844 $  25.30 $     *
        * $  188.9014716 $           * $  25.95 $  0.20
>>> example2.delete(1,2)
>>> print example2
@
@ Some objects in the GOODS field
@
unknown   $  189.2207323 $  62.2357983 $  26.87 $  0.32
   star   $  189.1409453 $  62.1696844 $  25.30 $     *
        * $  188.9014716 $           * $  25.95 $  0.20
\end{verbatim}
\end{enumerate}

\subsubsection{AsciiData method find()}
\label{adm_find}
\index{methods}\index{methods!AsciiData!find()}
\index{find()}
The method determines the column number for a given column name.
The value -1 is returned if a column with this name does not exist.

\prgrf{Usage}
adata\_obj.find(col\_name)

\prgrf{Parameters}
\begin{tabular}{lcl}
col\_name &{\it string}& the name of the column\\
\end{tabular}

\prgrf{Return}
- the column number or -1 if the column does not exist

\prgrf{Examples}
\begin{enumerate}
\item Search for the column with name 'column3':
\begin{verbatim}
>>> example2 = asciidata.open('example2.txt', null='*', \
                              delimiter='$', comment='@')
>>> cnum = example2.find('column2')
>>> cnum
1
>>>
\end{verbatim}
\item Search for the column with the name 'not\_there':
\begin{verbatim}
>>> example2 = asciidata.open('example2.txt', null='*', \
                              delimiter='$', comment='@')
>>> cnum = example2.find('not_there')
>>> cnum
-1
>>>
\end{verbatim}
Obviously the \ad object example2 does not have a column with this name.
\end{enumerate}

\subsubsection{AsciiData method flush()}
\label{adm_flush}
\index{methods}\index{methods!AsciiData!flush()}
\index{flush()}
The method updates the associated file with the newest version of the
\ad object.

\prgrf{Usage}
adata\_obj.flush()

\prgrf{Parameters}
-

\prgrf{Return}
-

\prgrf{Examples}
\begin{enumerate}
\item Manipulate an \ad object and update the file:
\begin{small}
\begin{verbatim}
work>more example.txt
#
# Some objects in the GOODS field
#
unknown  189.2207323  62.2357983  26.87  0.32
 galaxy  189.1408929  62.2376331  24.97  0.15
   star  189.1409453  62.1696844  25.30  0.12
 galaxy  188.9014716  62.2037839  25.95  0.20
work>python
Python 2.4.2 (#5, Oct 21 2005, 11:12:03)
[GCC 3.3.2] on sunos5
Type "help", "copyright", "credits" or "license" for more information.
>>> import asciidata
>>> example = asciidata.open('example.txt')
>>> del example[4]
>>> example.flush()
>>>
work>more example.txt
#
# Some objects in the GOODS field
#
unknown  189.2207323  62.2357983  26.87
 galaxy  189.1408929  62.2376331  24.97
   star  189.1409453  62.1696844  25.30
 galaxy  188.9014716  62.2037839  25.95
\end{verbatim}
\end{small}
\end{enumerate}

\subsubsection{AsciiData method info()}
\label{adm_info}
\index{methods}\index{methods!AsciiData!info()}
\index{info()}
The method returns an informative overview on the \ad object as
a string. This overview gives the user a quick insight
into e.g. the column names of the object. A further use of the
information within programmes is {\bf not} recommended, since all information
can also be retrieved by other in a machine usable format using
other methods. The overview contains:
\begin{itemize}
\item the name of the file associated to the \ad object;
\item the number of columns;
\item the number of rows;
\item the delimiter to separate columns;
\item the representing Null-values;
\item the comment string.
\end{itemize}
In addition, for every column the column name, type, format and
Null-representation is given.

\prgrf{Usage}
adata\_object.info()

\prgrf{Parameters}
-

\prgrf{Return}
-

\prgrf{Examples}
\begin{enumerate}
\item Print the information on an \ad object onto the screen:
\begin{small}
\begin{verbatim}
>>> example = asciidata.open('example.txt')
>>> print example.info()
File:       example.txt
Ncols:      4
Nrows:      4
Delimiter:  None
Null value: ['Null', 'NULL', 'None', '*']
Comment:    #
Column name:        column1
Column type:        <type 'str'>
Column format:      ['% 7s', '%7s']
Column null value : ['Null']
Column name:        column2
Column type:        <type 'float'>
Column format:      ['% 11.7f', '%12s']
Column null value : ['Null']
Column name:        column3
Column type:        <type 'float'>
Column format:      ['% 10.7f', '%11s']
Column null value : ['Null']
Column name:        column4
Column type:        <type 'float'>
Column format:      ['% 5.2f', '%6s']
Column null value : ['Null']
\end{verbatim}
\end{small}
\end{enumerate}

\subsubsection{AsciiData method insert()}
\label{adm_insert}
\index{methods}\index{methods!AsciiData!insert()}
\index{insert()}
This method inserts rows into all columns of the \ad object.
The second parameter controls where exactly the new, empty rows
are positioned. The number specified there the first empty
row will be .

\prgrf{Usage}
adata\_object.insert(nrows, start)

\prgrf{Parameters}
\begin{tabular}{lcl}
nrows &{\it int}& number of rows to be inserted\\
start &{\it int}& index position of the first inserted column\\
\end{tabular}

\prgrf{Return}
-

\prgrf{Examples}
\begin{enumerate}
\item Insert two rows such that the first row will have the index 1:
\begin{small}
\begin{verbatim}
>>> print example2
@
@ Some objects in the GOODS field
@
unknown   $  189.2207323 $  62.2357983 $  26.87 $  0.32
 galaxy   $            * $  62.2376331 $  24.97 $  0.15
   star   $  189.1409453 $  62.1696844 $  25.30 $     *
        * $  188.9014716 $           * $  25.95 $  0.20
>>> example2.insert(2,1)
>>> print example2
@
@ Some objects in the GOODS field
@
unknown   $  189.2207323 $  62.2357983 $  26.87 $  0.32
        * $            * $           * $      * $     *
        * $            * $           * $      * $     *
 galaxy   $            * $  62.2376331 $  24.97 $  0.15
   star   $  189.1409453 $  62.1696844 $  25.30 $     *
        * $  188.9014716 $           * $  25.95 $  0.20
\end{verbatim}
\end{small}
\end{enumerate}

\subsubsection{AsciiData method newcomment()}
\label{adm_newcomment}
\index{methods}\index{methods!AsciiData!newcomment()}
\index{newcomment()}
The method defines a new comment string for an \ad object.

\prgrf{Usage}
adata\_object.newcomment(comment)

\prgrf{Parameters}
\begin{tabular}{lcl}
comment &{\it string}& the string to indicate a comment\\
\end{tabular}

\prgrf{Return}
-

\prgrf{Examples}
\begin{enumerate}
\item Change the comment sign from '@' to '!':
\begin{small}
\begin{verbatim}
>>> print example2
@
@ Some objects in the GOODS field
@
unknown   $  189.2207323 $  62.2357983 $  26.87 $  0.32
 galaxy   $            * $  62.2376331 $  24.97 $  0.15
   star   $  189.1409453 $  62.1696844 $  25.30 $     *
        * $  188.9014716 $           * $  25.95 $  0.20
>>> example2.newcomment('!!')
>>> print example2
!!
!! Some objects in the GOODS field
!!
unknown   $  189.2207323 $  62.2357983 $  26.87 $  0.32
 galaxy   $            * $  62.2376331 $  24.97 $  0.15
   star   $  189.1409453 $  62.1696844 $  25.30 $     *
        * $  188.9014716 $           * $  25.95 $  0.20
\end{verbatim}
\end{small}
\end{enumerate}

\subsubsection{AsciiData method newdelimiter()}
\label{adm_newdelimiter}
\index{methods}\index{methods!AsciiData!newdelimiter()}
\index{newdelimiter()}
This method specifies a new delimiter for an \ad object.

\prgrf{Usage}
adata\_object.newdelimiter(delimiter)

\prgrf{Parameters}
\begin{tabular}{lcl}
delimiter &{\it string}& the new delimiter to separate columns\\
\end{tabular}

\prgrf{Return}
-

\prgrf{Examples}
\begin{enumerate}
\item Change the delimiter sign from '\$' to '\verb+<>+':
\begin{small}
\begin{verbatim}
>>> print example2
!!
!! Some objects in the GOODS field
!!
unknown   $  189.2207323 $  62.2357983 $  26.87 $  0.32
 galaxy   $            * $  62.2376331 $  24.97 $  0.15
   star   $  189.1409453 $  62.1696844 $  25.30 $     *
        * $  188.9014716 $           * $  25.95 $  0.20
>>> example2.newdelimiter('<>')
>>> print example2
!!
!! Some objects in the GOODS field
!!
unknown   <>  189.2207323 <>  62.2357983 <>  26.87 <>  0.32
 galaxy   <>            * <>  62.2376331 <>  24.97 <>  0.15
   star   <>  189.1409453 <>  62.1696844 <>  25.30 <>     *
        * <>  188.9014716 <>           * <>  25.95 <>  0.20
\end{verbatim}
\end{small}
\end{enumerate}

\subsubsection{AsciiData method newnull()}
\label{adm_newnull}
\index{methods}\index{methods!AsciiData!newnull()}
\index{newnull()}
The method specifies a new string to represent Null-entries in an
\ad object.

\prgrf{Usage}
adata\_object.newnull(newnull)

\prgrf{Parameters}
\begin{tabular}{lcl}
newnull &{\it string}& the representation for Null-entries\\
\end{tabular}

\prgrf{Return}
-

\prgrf{Examples}
\begin{enumerate}
\item Change the Null representation from '*' to 'NaN':
\begin{small}
\begin{verbatim}
>>> print example2
@
@ Some objects in the GOODS field
@
unknown   $  189.2207323 $  62.2357983 $  26.87 $  0.32
 galaxy   $            * $  62.2376331 $  24.97 $  0.15
   star   $  189.1409453 $  62.1696844 $  25.30 $     *
        * $  188.9014716 $           * $  25.95 $  0.20
>>> example2.newnull('NaN')
>>> print example2
@
@ Some objects in the GOODS field
@
unknown   $  189.2207323 $  62.2357983 $  26.87 $  0.32
 galaxy   $          NaN $  62.2376331 $  24.97 $  0.15
   star   $  189.1409453 $  62.1696844 $  25.30 $   NaN
      NaN $  188.9014716 $         NaN $  25.95 $  0.20
\end{verbatim}
\end{small}
\end{enumerate}

\subsubsection{AsciiData method writeto()}
\label{adm_writeto}
\index{methods}\index{methods!AsciiData!writeto()}
\index{writeto()}
Write the \ad object to a file. The file name is given in a parameter.

\prgrf{Usage}
 adata\_object.writeto(filename)

\prgrf{Parameters}
\begin{tabular}{lcl}
filename &{\it string}& the filename to save the \ad object to\\
\end{tabular}

\prgrf{Return}
-

\prgrf{Examples}
\begin{enumerate}
\item Write an \ad object to the file 'newfile.txt':
\begin{small}
\begin{verbatim}
>>> print example2
@
@ Some objects in the GOODS field
@
unknown   $  189.2207323 $  62.2357983 $  26.87 $  0.32
 galaxy   $            * $  62.2376331 $  24.97 $  0.15
   star   $  189.1409453 $  62.1696844 $  25.30 $     *
        * $  188.9014716 $           * $  25.95 $  0.20
>>> example2.writeto('newfile.txt')
>>>
> more newfile.txt
@
@ Some objects in the GOODS field
@
unknown   $  189.2207323 $  62.2357983 $  26.87 $  0.32
 galaxy   $            * $  62.2376331 $  24.97 $  0.15
   star   $  189.1409453 $  62.1696844 $  25.30 $     *
        * $  188.9014716 $           * $  25.95 $  0.20
\end{verbatim}
\end{small}
\end{enumerate}

%
% Section the AsciiColumn class
%
\subsection{The AsciiColumn class}
\label{acclass}
\index{classes}\index{classes!AsciiColumn}
\index{AsciiColumn class}
The \ac class is the the second important class in the \AAD module.
The \ac manages all column related issues, which means that even the
actual data is stored in \ac objects. These \ac object are accessed
via the \ad object, either specifying the column name
(such as e.g.\ \verb+adata_object['diff1']+) or the column index
(such as e.g.\ \verb+adata_object[3]+).

\subsubsection{AsciiColumn data}
\label{acd}
\index{class data}\index{AsciiColumn!data}
\ac objects contain some information which is important to the user
and can be used in the processing. Although it is possible,
this class data should {\bf never} be changed directly by the user.
All book-keeping is done internally.

\prgrf{Data}
\begin{tabular}{lcl}
colname &{\it string}& file name associated to the object\\
\end{tabular}


\subsubsection{AsciiColumn method get}
\label{acm_get}
\index{methods}\index{methods!AsciiColumn!get}
\index{get}
%
This method retrieves one list element of an \ac instance.
The element is specified with the row number.

\prgrf{Usage}
elem = acol\_object[row]\\
{\it or}\\
elem = operator.getitem(acol\_object, row)

\prgrf{Parameters}
\begin{tabular}{lcl}
row &{\it int}& the row number of the entry to be replaced\\
\end{tabular}


\prgrf{Return}
- the requested column element

\prgrf{Examples}
\begin{enumerate}
\item Retrieve and print the first element of the \ac instance which is the
third column of the \ad instance {\tt 'exa'}:
\begin{small}
\begin{verbatim}
>>> exa = asciidata.open('some_objects.cat')
>>> print exa
#
# most important objects
#
    1  1.0  red  23.08932 -19.34509
    2  9.5 blue  23.59312 -19.94546
    3  3.5 blue  23.19843 -19.23571
>>> elem = exa[2][0]
>>> print elem
 red
>>>
\end{verbatim}
\end{small}
\end{enumerate}

\subsubsection{AsciiData method set}
\label{acm_set}
\index{methods}\index{methods!AsciiColumn!set}
\index{set}
%
This methods sets list members, which means elements, of an \ac instance.
The list member to be changed is addressed via their row number.

\prgrf{Usage}
acol\_object[row] = an\_entry\\
{\it or}\\
operator.setitem(acol\_object, row, adata\_column)

\prgrf{Parameters}
\begin{tabular}{lcl}
row &{\it int}& the row number of the entry to be replaced\\
an\_entry &{\it string/integer/float}& the data to replace the previous entry\\
\end{tabular}
%

\prgrf{Return}
-

\prgrf{Examples}
\begin{enumerate}
\item Replace the third entry of the column which is the third column
in the \ad instance {\tt 'exa'}:
\begin{small}
\begin{verbatim}
>>> print exa
#
# most important objects
#
    1  1.0  red  23.08932 -19.34509
    2  9.5 blue  23.59312 -19.94546
    3  3.5 blue  23.19843 -19.23571
>>> exa[2][2] = 'green'
>>> print exa
#
# most important objects
#
    1  1.0  red  23.08932 -19.34509
    2  9.5 blue  23.59312 -19.94546
    3  3.5 green  23.19843 -19.23571
>>>
\end{verbatim}
\end{small}
\end{enumerate}


\subsubsection{AsciiColumn method len()}
\label{acm_len}
\index{methods}\index{methods!AsciiColumn!len()}
\index{len()}
%
This method defines a length of an \ac instance, which euqals the
number of row ins the \ac.

\prgrf{Usage}
len(ac\_object)


\prgrf{Parameters}
-

\prgrf{Return}
- the length (= number of rows) of the \ac

\prgrf{Examples}
\begin{enumerate}
\item Print the length of the fifth column onto the screen:
\begin{small}
\begin{verbatim}
>>> exa = asciidata.open('some_objects.cat')
>>> print exa
#
# most important objects
#
    1  1.0  red  23.08932 -19.34509
    2  9.5 blue  23.59312 -19.94546
    3  3.5 blue  23.19843 -19.23571
>>> print len(exa[4])
3
>>>
\end{verbatim}
\end{small}
\end{enumerate}


\subsubsection{AsciiColumn iterator type}
\label{acm_iterator}
\index{methods}\index{methods!AsciiColumn!iterator}
\index{iterator}
%
This defines an iterator over an \ac instance. The iteration is finished after
{\tt acolumn\_object.nrows} calls and returns each element in subsequent calls.
Please not that it is {\bf not} possible to change these elements.

\prgrf{Usage}
for element in acolumn\_object:\\
... $<do\ something>$

\prgrf{Parameters}
-

\prgrf{Return}
-

\prgrf{Examples}
\begin{enumerate}
\item Iterate over an \ac instance and print the elements:
\begin{small}
\begin{verbatim}
>>> print exa
#
# most important objects
#
    1  1.0  red  23.08932 -19.34509
    2  9.5 blue  23.59312 -19.94546
    3  3.5 blue  23.19843 -19.23571
>>> acol = exa[4]
>>> for element in acol:
...     print element
...
-19.34509
-19.94546
-19.23571
>>>
\end{verbatim}
\end{small}
\end{enumerate}



\subsubsection{AsciiColumn method copy()}
\label{acm_copy}
\index{methods}\index{methods!AsciiColumn!copy()}
\index{copy()}
This method generates a so-called {\it deep copy} \index{deep copy}
of a column. This means the copy is not only a reference to an
existing column, but a real copy with all data.

\prgrf{Usage}
adata\_object[colname].copy()


\prgrf{Parameters}
-

\prgrf{Return}
- the copy of the column

\prgrf{Examples}
\begin{enumerate}
\item Copy the column 5 of \ad object 'example2' to column 2 of \ad
object 'example1'
\begin{small}
\begin{verbatim}
>>> print example1
#
# Some objects in the GOODS field
#
unknown  189.2207323  62.2357983  26.87  0.32
 galaxy            *  62.2376331  24.97  0.15
   star  189.1409453  62.1696844  25.30  0.12
 galaxy  188.9014716  62.2037839  25.95  0.20
>>> print example2
@
@ Some objects in the GOODS field
@
unknown   $  189.2207323 $  62.2357983 $  26.87 $  0.32
 galaxy   $            * $  62.2376331 $  24.97 $  0.15
   star   $  189.1409453 $  62.1696844 $  25.30 $     *
        * $  188.9014716 $           * $  25.95 $  0.20
>>> example1[1] = example2[4].copy()
>>> print example1
#
# Some objects in the GOODS field
#
unknown  0.32  62.2357983  26.87  0.32
 galaxy  0.15  62.2376331  24.97  0.15
   star     *  62.1696844  25.30  0.12
 galaxy  0.20  62.2037839  25.95  0.20
\end{verbatim}
\end{small}
\end{enumerate}

\subsubsection{AsciiColumn method get\_format()}
\label{acm_getformat}
\index{methods}\index{methods!AsciiColumn!get\_format()}
\index{get\_format()}
The method returns the format of the \ac object
The format description in \AAD is taken from Python. The Python Library
Reference
(\htmladdnormallink{Chapt.\ 2.3.6.2 in Python 2.4 }{http://www.python.org/doc/2.4.2/lib/typesseq-strings.html\#l2h-211})
gives a list of all possible formats.

\prgrf{Usage}
adata\_object[colname].get\_format()

\prgrf{Parameters}
-

\prgrf{Return}
- the format of the \ac object

\prgrf{Examples}
\begin{enumerate}
\item Get the format of \ac 0:
\begin{small}
\begin{verbatim}
>>> print example2
@
@ Some objects in the GOODS field
@
unknown   $  189.2207323 $  62.2357983 $  26.87 $  0.32
 galaxy   $            * $  62.2376331 $  24.97 $  0.15
   star   $  189.1409453 $  62.1696844 $  25.30 $     *
        * $  188.9014716 $           * $  25.95 $  0.20
>>> example2[0].get_format()
'% 9s'
\end{verbatim}
\end{small}
\end{enumerate}

\subsubsection{AsciiColumn method get\_type()}
\label{acm_gettype}
\index{methods}\index{methods!AsciiColumn!get\_type()}
\index{get\_type()}
The method returns the type of an \ac object

\prgrf{Usage}
adata\_object[colname].get\_type()

\prgrf{Parameters}
-

\prgrf{Return}
- the type of the \ac

\prgrf{Examples}
\begin{enumerate}
\item Get the type of \ac 0:
\begin{small}
\begin{verbatim}
>>> print example2
@
@ Some objects in the GOODS field
@
unknown   $  189.2207323 $  62.2357983 $  26.87 $  0.32
 galaxy   $            * $  62.2376331 $  24.97 $  0.15
   star   $  189.1409453 $  62.1696844 $  25.30 $     *
        * $  188.9014716 $           * $  25.95 $  0.20
>>> example2[0].get_type()
<type 'str'>
\end{verbatim}
\end{small}
\end{enumerate}


\subsubsection{AsciiColumn method get\_nrows()}
\label{acm_getnrows}
\index{methods}\index{methods!AsciiColumn!get\_nrows()}
\index{get\_nrows()}
This method offers a way to derive the number of rows in a \ac instance.

\prgrf{Usage}
acolumn\_object.get\_nrows()

\prgrf{Parameters}
-

\prgrf{Return}
- the number of rows

\prgrf{Examples}
\begin{enumerate}
\item get the number of rows in the column named 'column1':
\begin{small}
\begin{verbatim}
>>> exa = asciidata.open('sort_objects.cat')
>>> exa['column1'].get_nrows()
12
>>> print exa
    1     0     1     1
    2     1     0     3
    3     1     2     4
    4     0     0     2
    5     1     2     1
    6     0     0     3
    7     0     2     4
    8     1     1     2
    9     0     1     5
   10     1     2     6
   11     0     0     6
   12     1     1     5
>>>
\end{verbatim}
\end{small}
\end{enumerate}

\subsubsection{AsciiColumn method get\_unit()}
\label{acm_getunit}
\index{methods}\index{methods!AsciiColumn!get\_unit()}
\index{get\_unit()}
%
The method returns the unit of an \ac instance.

\prgrf{Usage}
acolumn\_object.get\_unit()

\prgrf{Parameters}
-

\prgrf{Return}
- the unit of the column

\prgrf{Examples}
\begin{enumerate}
\item Print the overview of the \ac with index 1:
\begin{small}
\begin{verbatim}
test>more some_objects.cat
# 1 NUMBER          Running object number
# 2 X_Y
# 3 COLOUR
# 4 RA              Barycenter position along world x axis          [deg]
# 5 DEC             Barycenter position along world y axis          [deg]
#
# most important objects
#
1 1.0  red 23.08932 -19.34509
2 9.5 blue 23.59312 -19.94546
3 3.5 blue 23.19843 -19.23571
test>python
Python 2.4.2 (#1, Nov 10 2005, 11:34:38)
[GCC 3.3.3 20040412 (Red Hat Linux 3.3.3-7)] on linux2
Type "help", "copyright", "credits" or "license" for more information.
>>> import asciidata
>>> exa = asciidata.open('some_objects.cat')
>>> print exa['RA'].get_unit()
deg
>>>
\end{verbatim}
\end{small}
\end{enumerate}


\subsubsection{AsciiColumn method info()}
\label{acm_info}
\index{methods}\index{methods!AsciiColumn!info()}
\index{info()}
The method gives an overview on an \ac object including its type,
format and the number of elements.

\prgrf{Usage}
adata\_object[colname].info()

\prgrf{Parameters}
-

\prgrf{Return}
- the overview on the \ac object

\prgrf{Examples}
\begin{enumerate}
\item Print the overview of the \ac with index 1:
\begin{small}
\begin{verbatim}
>>> print example2
@
@ Some objects in the GOODS field
@
unknown   $  189.2207323 $  62.2357983 $  26.87 $  0.32
 galaxy   $            * $  62.2376331 $  24.97 $  0.15
   star   $  189.1409453 $  62.1696844 $  25.30 $     *
        * $  188.9014716 $           * $  25.95 $  0.20
>>> print example2[1].info()
Column name:        column2
Column type:        <type 'float'>
Column format:      ['% 11.7f', '%12s']
Column null value : ['*']
\end{verbatim}
\end{small}
\end{enumerate}

\subsubsection{AsciiColumn method reformat()}
\label{acm_reformat}
\index{methods}\index{methods!AsciiColumn!reformat()}
\index{reformat()}
The method gives a new format\index{format} to an \ac object. Please note
that the new format does {\bf not} change the column content,
but only the string representation of the content.
The format description in \AAD is taken from Python. The Python Library
Reference
(\htmladdnormallink{Chapt.\ 2.3.6.2 in Python 2.4 }{http://www.python.org/doc/2.4.2/lib/typesseq-strings.html\#l2h-211})
gives a list of all possible formats.


\prgrf{Usage}
adata\_object[colname].reformat('newformat')

\prgrf{Parameters}
\begin{tabular}{lcl}
new\_format &{\it string}& the new format of the \ac \\
\end{tabular}

\prgrf{Return}
-

\prgrf{Examples}
\begin{enumerate}
\item Change the format of the \ac with index 1:
\begin{small}
\begin{verbatim}
>>> print example2
@
@ Some objects in the GOODS field
@
unknown   $  189.2207323 $  62.2357983 $  26.87 $  0.32
 galaxy   $            * $  62.2376331 $  24.97 $  0.15
   star   $  189.1409453 $  62.1696844 $  25.30 $     *
        * $  188.9014716 $           * $  25.95 $  0.20
>>> example2[1].reformat('% 6.2f')
>>> print example2
@
@ Some objects in the GOODS field
@
unknown   $  189.22 $  62.2357983 $  26.87 $  0.32
 galaxy   $       * $  62.2376331 $  24.97 $  0.15
   star   $  189.14 $  62.1696844 $  25.30 $     *
        * $  188.90 $           * $  25.95 $  0.20
\end{verbatim}
\end{small}
\end{enumerate}

\subsubsection{AsciiColumn method rename()}
\label{acm_rename}
\index{methods}\index{methods!AsciiColumn!rename()}
\index{rename()}
The method changes the name on \ac object.

\prgrf{Usage}
adata\_object[colname].rename('newname')

\prgrf{Parameters}
\begin{tabular}{lcl}
newname &{\it string}& the filename to save the \ad object to\\
\end{tabular}

\prgrf{Return}
-

\prgrf{Examples}
\begin{enumerate}
\item Change the column name from 'column1' to 'newname':
\begin{small}
\begin{verbatim}
>>> print example2[3].info()
Column name:        column4
Column type:        <type 'float'>
Column format:      ['% 5.2f', '%6s']
Column null value : ['*']

>>> example2[3].rename('newname')
>>> print example2[3].info()
Column name:        newname
Column type:        <type 'float'>
Column format:      ['% 5.2f', '%6s']
Column null value : ['*']
\end{verbatim}
\end{small}
\end{enumerate}

\subsubsection{AsciiColumn method tonumarray()}
\label{acm_tonumarray}
\index{methods}\index{methods!AsciiColumn!tonumarray()}
\index{tonumarray()}
The method converts the content of an \ad object into a numarray
object. Note that this is only possible if there are no Null-entries
in the column, since numarray would not allow these Null-entries.

\prgrf{Usage}
adata\_object[colname].tonumarray()

\prgrf{Parameters}
-

\prgrf{Return}
- the \ac content in a numarray object.

\prgrf{Examples}
\begin{enumerate}
\item Change the column name from 'column1' to 'newname':
\begin{small}
\begin{verbatim}
>>> print example2
@
@ Some objects in the GOODS field
@
unknown   $  189.2207323 $  62.2357983 $  26.87 $  0.32
 galaxy   $            * $  62.2376331 $  24.97 $  0.15
   star   $  189.1409453 $  62.1696844 $  25.30 $     *
        * $  188.9014716 $           * $  25.95 $  0.20
>>> numarr =  example2[3].tonumarray()
>>> numarr
array([ 26.87,  24.97,  25.3 ,  25.95])
\end{verbatim}
\end{small}
\end{enumerate}



%
% Section the Header class
%
\subsection{The Header class}
\label{hclass}
\index{classes}\index{classes!Header}
\index{Header class}
The Header class manages the header of an \ad object.
The header contains a list of comments.
Any kind of meta-data such as column names are part of the
columns and therefore the AsciiColumn (see Sect.\ \ref{acclass})
class. The header object is accessed through various
methods to e.g. get or set items.

\subsubsection{Header method get}
\label{ahe_get}
\index{methods}\index{methods!Header!get}
\index{get}
%
The header class contains a method to get individual items from a header
instance via their index.

\prgrf{Usage}
header\_entry = adata\_object.header[index]\\
{\it or}\\
header\_entry = operator.getitem(adata\_object.header, index)

\prgrf{Parameters}
\begin{tabular}{lcl}
index &{\it int}& the index of the item to retrieve\\
\end{tabular}

\prgrf{Return}
- one entry of the header

\prgrf{Examples}
\begin{enumerate}
\item Retrieve the second entry of this table header:
\begin{small}
\begin{verbatim}
>>> print example
#
# most important sources!!
#
    1  1.0  red  23.08932 -19.34509
    2  9.5 blue  23.59312 -19.94546
    3  3.5 blue  23.19843 -19.23571
>>> header_entry = example.header[1]
>>> print header_entry
 most important sources!!

>>>
\end{verbatim}
\end{small}
\item Access the third entry of this table header:
\begin{small}
\begin{verbatim}
>>> print example
#
# most important sources!!
#
    1  1.0  red  23.08932 -19.34509
    2  9.5 blue  23.59312 -19.94546
    3  3.5 blue  23.19843 -19.23571
>>> example.header[2]
'\n'
>>>
\end{verbatim}
\end{small}
\end{enumerate}


\subsubsection{Header method set}
\label{ahe_set}
\index{methods}\index{methods!Header!set}
\index{set}
%
The header class contains a method to set individual items in a header.
The item is specified via its index.

\prgrf{Usage}
adata\_object.header[index] = new\_entry\\
{\it or}\\
header\_entry = operator.setitem(adata\_object.header, index, new\_entry)

\prgrf{Parameters}
\begin{tabular}{lcl}
index &{\it int}& the index of the item to be set\\
new\_entry &{\it string}& the new content of the header item\\
\end{tabular}

\prgrf{Return}
-

\prgrf{Examples}
\begin{enumerate}
\item Change the second header item:
\begin{small}
\begin{verbatim}
>>> print example
#
# most important sources!!
#
    1  1.0  red  23.08932 -19.34509
    2  9.5 blue  23.59312 -19.94546
    3  3.5 blue  23.19843 -19.23571
>>> example.header[1] = 'a new header entry?'
>>> print example
#
#a new header entry?
#
    1  1.0  red  23.08932 -19.34509
    2  9.5 blue  23.59312 -19.94546
    3  3.5 blue  23.19843 -19.23571
>>>
\end{verbatim}
\end{small}
\item Change the third header item:
\begin{small}
\begin{verbatim}
>>> print example
#
# most important sources!!
#
    1  1.0  red  23.08932 -19.34509
    2  9.5 blue  23.59312 -19.94546
    3  3.5 blue  23.19843 -19.23571
>>> example.header[2] = '   >>> dont forget leading spaces if desired!'
>>> print example
#
# most important sources!!
#   >>> dont forget leading spaces if desired!
    1  1.0  red  23.08932 -19.34509
    2  9.5 blue  23.59312 -19.94546
    3  3.5 blue  23.19843 -19.23571
>>>
\end{verbatim}
\end{small}
\end{enumerate}


\subsubsection{Header method del}
\label{ahe_del}
\index{methods}\index{methods!Header!del}
\index{del}
%
The header class contains a method to delete individual items in a header.
The item is specified via its index.

\prgrf{Usage}
del adata\_object.header[index]\\
{\it or}\\
operator.delitem(adata\_object.header, index)

\prgrf{Parameters}
\begin{tabular}{lcl}
index &{\it int}& the index of the item to be deleted\\
\end{tabular}

\prgrf{Return}
-

\prgrf{Examples}
\begin{enumerate}
\item Delete the second header item:
\begin{small}
\begin{verbatim}
>>> print example
#
# most important sources!!
#
    1  1.0  red  23.08932 -19.34509
    2  9.5 blue  23.59312 -19.94546
    3  3.5 blue  23.19843 -19.23571
>>> del example.header[1]
>>> print example
#
#
    1  1.0  red  23.08932 -19.34509
    2  9.5 blue  23.59312 -19.94546
    3  3.5 blue  23.19843 -19.23571
>>>
\end{verbatim}
\end{small}
\end{enumerate}

\subsubsection{Header method str()}
\label{ahe_str}
\index{methods}\index{methods!Header!str()}
\index{str()}
%
This method converts the entire \ah instance into a string. The {\tt print}
command called with an \ah instance as first parameter also prints the
string created using this method {\tt str()}.

\prgrf{Usage}
str(adata\_object.header)

\prgrf{Parameters}
-

\prgrf{Return}
- the string representation of the \ah instance

\prgrf{Examples}
\begin{enumerate}
\item Delete the second header item:
\begin{small}
\begin{verbatim}
>>> print example
#
# most important sources!!
#
    1  1.0  red  23.08932 -19.34509
    2  9.5 blue  23.59312 -19.94546
    3  3.5 blue  23.19843 -19.23571
>>> print example.header
#
# most important sources!!
#

>>>
\end{verbatim}
\end{small}
\end{enumerate}

\subsubsection{Header method len()}
\label{ahe_len}
\index{methods}\index{methods!Header!len()}
\index{len()}
%
The method defines the length of an \ah instance, which
equals the number of the comment entries. Please note that
also empty lines are are counted.

\prgrf{Usage}
len(adata\_object.header)

\prgrf{Parameters}
-

\prgrf{Return}
- the length of the \ah instance

\prgrf{Examples}
\begin{enumerate}
\item Get the length of an \ah:
\begin{small}
\begin{verbatim}
>>> print example
#
# most important sources!!
#
    1  1.0  red  23.08932 -19.34509
    2  9.5 blue  23.59312 -19.94546
    3  3.5 blue  23.19843 -19.23571
>>> header_length = len(example.header)
>>> header_length
3
>>>
\end{verbatim}
\end{small}
\end{enumerate}


\subsubsection{Header method reset()}
\label{ahe_res}
\index{methods}\index{methods!Header!reset()}
\index{reset()}
%
The method deletes all entries from an \ah instance and provides
a clean, empty header.

\prgrf{Usage}
adata\_object.header.reset()

\prgrf{Parameters}
-

\prgrf{Return}
-

\prgrf{Examples}
\begin{enumerate}
\item Reset an \ah instance:
\begin{small}
\begin{verbatim}
>>> print example
#
# most important sources!!
#
    1  1.0  red  23.08932 -19.34509
    2  9.5 blue  23.59312 -19.94546
    3  3.5 blue  23.19843 -19.23571
>>> example.header.reset()
>>> print example
    1  1.0  red  23.08932 -19.34509
    2  9.5 blue  23.59312 -19.94546
    3  3.5 blue  23.19843 -19.23571
>>>
\end{verbatim}
\end{small}
\end{enumerate}

\subsubsection{Header method append()}
\label{ahe_append}
\index{methods}\index{methods!Header!append()}
\index{append()}
The method appends a string or a list of strings to the header
of an \ad object.

\prgrf{Usage}
adata\_object.header.append(hlist)

\prgrf{Parameters}
\begin{tabular}{lcl}
hlist &{\it string}& the list of strings to be appended to the header\\
\end{tabular}

\prgrf{Return}
-

\prgrf{Examples}
\begin{enumerate}
\item Change the column name from 'column1' to 'newname':
\begin{small}
\begin{verbatim}
>>> print example2
@
@ Some objects in the GOODS field
@
unknown   $  189.2207323 $  62.2357983 $  26.87 $  0.32
 galaxy   $            * $  62.2376331 $  24.97 $  0.15
   star   $  189.1409453 $  62.1696844 $  25.30 $     *
        * $  188.9014716 $           * $  25.95 $  0.20
>>> example2.header.append('Now a header line is appended!')
>>> print example2
@
@ Some objects in the GOODS field
@
@ Now a header line is appended!
unknown   $  189.2207323 $  62.2357983 $  26.87 $  0.32
 galaxy   $            * $  62.2376331 $  24.97 $  0.15
   star   $  189.1409453 $  62.1696844 $  25.30 $     *
        * $  188.9014716 $           * $  25.95 $  0.20
>>> example2.header.append("""And now we try to put
... even a set of lines
... into the header!!""")
>>> print example2
@
@ Some objects in the GOODS field
@
@ Now a header line is appended!
@ And now we try to put
@ even a set of lines
@ into the header!!
unknown   $  189.2207323 $  62.2357983 $  26.87 $  0.32
 galaxy   $            * $  62.2376331 $  24.97 $  0.15
   star   $  189.1409453 $  62.1696844 $  25.30 $     *
        * $  188.9014716 $           * $  25.95 $  0.20
\end{verbatim}
\end{small}
\end{enumerate}
